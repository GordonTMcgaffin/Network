% File:    report.tex
% Brief:   report for CS 354 Project 2
% Author:  G. Mcgaffin (23565608@sun.ac.za)
% Author:  J. P. Visser (21553416@sun.ac.za)
% Date:    2022-09-02

\documentclass[10pt, a4paper]{article}

\usepackage{amsmath}
\usepackage{booktabs}
\usepackage{graphicx}
%\graphicspath{ {./images/exp1/}{./images/exp2/}{./images/exp3/} }

\title{CS 354 Project 3: Network Address Translation}
\author{Group 41: \vspace{0.5em} \\
        G. Mcgaffin (23565608@sun.ac.za) \vspace{0.3em} \\
        J.\ P.\ Visser (21553416@sun.ac.za)}
\date{\vspace{1em} 2 September 2022}

\begin{document}

% --- Title ------------------------------------------------------------------ %

\maketitle
\newpage

% --- Table of Contents ------------------------------------------------------ %

\tableofcontents
\newpage

% --- Introduction ----------------------------------------------------------- %

\section{Introduction}
\label{sec:intro}

For this project we were tasked with implementing a simulated router with
minimal NAT functionality, as well as a simple client.

The router processes paquets received from clients, who are either marked as
internal (hosts in the local network) or external (hosts belonging to an
external network).

Clients send simple paquets to one another, and the router has to process these
paquets in an appropriate manner, depending on the following cases:
\begin{itemize}
  \item \textbf{Internal $\rightarrow$ Internal}: The paquet is forwarded
  without changing its data
  \item \textbf{Internal $\rightarrow$ External}: The paquet header is modified
  and an entry is added to the NAT table, or refreshed if it already exists,
  that binds the source IP/port to the destination address.
  \item \textbf{External $\rightarrow$ Internal}: The paquet is routed according
  to the entry in the NAT table. If there is no corresponding entry in the table
  for the source, the paquet is dropped and an error paquet is returned (unless
  something like port forwarding has been implemented).
  \item \textbf{External $\rightarrow$ External}: Paquets are dropped, as they
  should be routed by external networks.
\end{itemize}

Additional features we needed to implement were:

\begin{itemize}
  \item \textbf{NAT table}. The address translation table should refresh
    dynamically.
  \item \textbf{DHCP}. Internal clients should automatically request and be
    assigned an address by the NAT server.
\end{itemize}

\subsection{Overview}
\label{subsec:oview}

In this document we will provide a complete view of our implementation, by
discussing its design (\S\ref{sec:design}), giving a breakdown of the files into
which it is organized (\S\ref{sec:fdesc}), and providing a high level
description, providing more details where necessary, of the flow of execution of
the two programs which it comprises (\S\ref{sec:pdesc}).

We are confident that our implementation meets all of the requirements. However,
we did not implement any features beyond those listed in the requirements
either. Therefore, we do not include sections dedicated to unimplemented or
additional features, as it would be superfluous.

Furthermore, we will discuss issues we encountered during the development
process (\S\ref{sec:issues}), experiments we have conducted (\S\ref{sec:exp}),
compilation and execution of the router and client (\S\ref{sec:comp} and
\S\ref{sec:exec}), as well as the libraries we made use of (\S\ref{sec:libs}).

% --- Additional Features ---------------------------------------------------- %

% --- File Descriptions ------------------------------------------------------ %

\section{File Descriptions}
\label{sec:fdesc}

% --- Program Description ---------------------------------------------------- %

\section{Program Description}
\label{sec:pdesc}

% --- Experiments ------------------------------------------------------------ %

\section{Experiments}
\label{sec:exp}


% --- Issues Encountered ----------------------------------------------------- %

\section{Issues Encountered}
\label{sec:issues}


% --- Design ----------------------------------------------------------------- %

\section{Design}
\label{sec:design}

% --- Compilation ------------------------------------------------------------ %

\section{Compilation}
\label{sec:comp}

It is assumed that the project will be run on Linux from a Bash shell.

\subsection{Dependencies}
\label{subsec:deps}

% --- Execution -------------------------------------------------------------- %

\section{Execution}
\label{sec:exec}

% --- Libraries -------------------------------------------------------------- %

\section{Libraries}
\label{sec:libs}

\end{document}
